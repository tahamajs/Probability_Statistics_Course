
استاد حواس پرتی را در نظر بگیرید که برای دو دانشجو در زمان یکسان قرار ملاقات می گذارد.اما متاسفانه استاد در هر زمان فقط می تواند با یک دانشجو ملاقات کند.مدت زمان ملاقات دو دانشجو مستقل از یکدیگر و دارای توزیع نمایی با میانگین 30 دقیقه است.امید ریاضی فاصله زمانی بین ورود دانشجوی اول و خروج دانشجوی دوم را در دو حالت زیر بیابید.
\subsection*{الف}
دانشجوی اول سر وقت حاضر می شود ولی دانشجوی دوم 5 دقیقه دیر میرسد.(تذکر:جواب 60 دقیقه یا 65 دقیقه نیست)
\subsection*{ب}
دانشجوی اول سر وقت حاضر می شود ولی دانشجوی دوم $X$ دقیقه دیر میرسد که $X$ دارای توزیع نمایی با میانگین 5 دقیقه است.

