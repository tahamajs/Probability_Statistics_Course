
در هر مورد تابع توزیع خواسته شده را به دست آورید و سپس تحقیق کنید متغیر تصادفی مورد نظر از چه خانواده ای از توزیع هاست. و با استفاده از آن امید ریاضی و واریانس توزیع را به دست آورید.
\subsection*{الف}
فرض کنید متغیر تصادفی $X$ از توزیع پارتو با متغیر $\theta > 0$ پیروی می کند که تابع توزیع آن به صورت زیر است:
$$f_{X}(x) = \frac{\theta}{x ^ {\theta + 1}} \quad x > 1$$
حال اگر متغیر تصادفی $Y$ به صورت زیر به دست بیاید، $Y$ از چه توزیعی پیروی می کند؟
$$ Y = \ln X$$

\subsection*{ب}
فرض کنید $Y$ متغیر تصادفی با توزیع نرمال استاندارد باشد. یعنی $Y \sim exponential(\lambda)$. حال اگر متغیر تصادفی $W$ به صورت زیر به دست بیاید، تابع توزیع آن را به دست بیاورید.$W$ از چه توزیعی پیروی می کند؟

$$W = \sqrt{Y}$$
\subsection*{ج}
فرض کنید $Z$ متغیر تصادفی با توزیع نرمال استاندارد باشد یعنی $Z \sim N(0, 1)$ . حال اگر متغیر تصادفی $Y$ به صورت زیر به دست بیاید، تابع توزیع آن را به دست بیاورید.$Y$ از چه توزیعی پیروی می کند؟
$$Z = e ^ Z$$
\subsection*{د}
فرض کنید $Z$ متغیر تصادفی با توزیع نرمال استاندارد باشد یعنی $Z \sim N(0, 1)$ . حال اگر متغیر تصادفی $X$ به صورت زیر به دست بیاید، تابع توزیع آن را به دست بیاورید.$X$ از چه توزیعی پیروی می کند؟
$$X = Z ^ 2 $$
