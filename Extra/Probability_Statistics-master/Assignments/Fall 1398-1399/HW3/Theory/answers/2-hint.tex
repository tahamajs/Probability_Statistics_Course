
تابع توزیع بتا با پارامتر های $a$ و $b$ از رابطه ی زیر به دست می آید:
$$f(x) = \frac{1}{B(a,b)}x^{a - 1}(1 - x)^{b - 1}$$
و هم چنین توجه کنید که مود های تابع در نقاط ماکسیم محلی رخ میدهد.در نتیجه باید از تابع توزیع مشتق بگیریم.

\subsection*{الف}
اگر $a> 1$ و $b > 1$ آنگاه:
$$f'(x) = \frac{1}{B(a,b)}((a - 1)x ^ {a - 2}(1 - x)^{b - 1} - (b - 1)x ^ {a - 1}(1 - x)^ { b - 2})$$
$$= \frac{1}{B(a,b)}x^{a - 2}(1 - x)^{b - 2}((a - 1) - (a + b - 2)x) = 0 $$
با بررسی عملکرد مشتق می توان فهمید که ماکسیمم در نقطه $x = \frac{a - 1}{a +b - 2}$ رخ می دهد
\subsection*{ب}
اگر $a = 1$ آنگاه:
$$f(x) = \frac{1}{B(1,b)}(1 - x)^{b - 1} \quad 0 \leq x \leq 1$$
در نتیجه مشتق آن برابر می شود با:
$$f(x) = \frac{1}{B(1,b)}(1 - b)(1 - x)^{b - 2} \quad 0 \leq x \leq 1$$
و این مقدار همیشه مثبت است در نتیجه تابع صعودی است.(در بازه 0 تا 1) \\
اگر $b = 1$ آنگاه:
$$f(x) = \frac{1}{B(a , 1)}x ^ {a - 1} \quad 0 \leq x \leq 1$$
در نتیجه مشتق آن برابر می شود با:
$$f(x) = \frac{1}{B(a , 1)}(a - 1)x ^ {a - 2} \quad 0 \leq x \leq 1$$
و این مقدار همیشه منفی است و در نتیجه تابع نزولی است.(در بازه 0 تا 1))\\
اگر $a$ و $b$ هر دو کمتر از یک باشند با بررسی دوباره رفتار مشتق ، به این نتیجه می رسیم که $f(x)$ در صورتی که $x$ کمتر از $\frac{a - 1}{a + b   2}$ باشد نزولی و اگر $x$ بیشتر از مقدار گفته شده باشد صعودی است در مقدار گفته شده مشتق صفر است.پس دو حالت پیش می آید.اگر $f(0) = f(1)$ آنگاه $f(x)$ به صورت $U-shaped$ است با مود در مقادیر 0 و 1. اگر هم مقدار $f(0)$ با $f(1)$ برابر نباشد، مود یا در 0 به وجود میاید یا در 1.

\subsection*{ج}
وقتی که $a = 1 = b$ آنگاه $B(1 , 1) = 1$ در نتیجه تابع توزیع برابر می شود با :
$$f(x) = \frac{1}{B(1,1)} = 1 \quad 0 < x < 1 $$
 در نتیجه همه نقاط در این بازه ، مد هستند.
