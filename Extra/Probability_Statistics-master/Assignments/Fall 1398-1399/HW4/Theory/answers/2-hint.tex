\subsection*{الف}
راه حل بخش اول  \\
 برای اینکه 
$kx^2$
تابع چگالی احتمال باشد باید داشته باشیم:
$$
\int_{0}^{3}f_X(x) \, dx = 1 \implies \int_{0}^{3}kx^2 \, dx = 1 \implies 9k = 1 \implies k = \dfrac{1}{9}
$$
$$
E[Y] = E[X^3] = \int_{0}^{3} x^3f_X(x) \, dx = \dfrac{1}{9} \int_{0}^{3} x^5 \, dx = \dfrac {27}{2}
$$
$$
E[Y^2] = E[X^6] = \int_{0}^{3}x^6f_X(x) \, dx = \dfrac{1}{9} \int_{0}^{3}x^8 \, dx = 3^5
$$
$$
var(Y) = E[Y^2] - E[Y]^2 = \dfrac{243}{4}
$$
\subsection*{ب}
راه حل بخش دوم  \\
$Z = \frac{U}{V}  ,   0 \leq z \leq 1  $ 
قرار میدهیم . حال ابتدا تابع توزیع تجمعی  
$F_Z(z)$
 را حساب میکنیم.
$$
F_Z(z)=P({\dfrac{U}{V}}  \leq z )  \cap  P( ( X \leq Y) \cup (X  > Y))
$$
 در تساوی بالا پرانتزی که با تعریف توزیع تجمعی اشتراک گرفته شده است شرط داشتن مینیمم برای توزیع های $X$ و $Y$ است .  حال با استفاده از خاصیت پخشی داریم :
$$
F_Z(z)= (P({\dfrac{U}{V}}  \leq z ) \cap  ( X \leq Y) ) \cup  (P({\dfrac{U}{V}}  \leq z ) \cap  ( X > Y) )
$$
سمت راست تساوی بالا را دقت کنید که در آن دو عبارت با هم اجتماع شده اند ، اگر به هرکدام از این دو عبارت دقت کنیم درمیابیم که اگر شرط های 
$X \leq Y$ و 
$X > Y$
 بخواهند برقرار باشند ، میتوان $U$ و $V$ را با $X$ و $Y$ جاگذاری کرد . بنابراین خواهیم داشت:
$$
F_Z(z)=P({\dfrac{X}{Y}} \leq z , X \leq Y) + P ( {\dfrac{Y}{X}} \leq z , X > Y)
$$
$$
F_Z(z)=P(X \leq Yz , X \leq Y ) + P( Y \leq Xz, X > Y)
$$
$$
F_Z(z)=\int_{0}^{\infty} \int_{0}^{yz}f_{XY}(x,y) \, dxdy + \int_{0}^{\infty} \int_{0}^{xz}f_{XY}(x,y) \, dydx
$$
در این جا از دو طرف نسبت به z مشتق میگیریم و داریم:
$$
f_Z(z) = \int_{0}^{\infty}yf_{XY}(x, y)dy + \int_{0}^{\infty}xf_{XY}(x, y)\, dx  
$$
با توجه به استقلال دو متغیر $X$ و $Y$ و توزیع آنها و محاسبه انتگرال فوق خواهیم داشت:
$$
  \implies f_Z(z) = \int_{0}^{\infty}ye^{-(yz + y)}dy + \int_{0}^{\infty}xe^{-(xz + x)}dx
$$
 با توجه به راهنمایی موجود در سوال داریم:
$$
\int_xe^{-ax}\, dx = -\dfrac{(ax+1)e^{-ax}}{a^2}
$$
$$
\implies f_Z{z} = 2\dfrac{1}{(z + 1)^2}
$$
$$
f_Z(z) = \begin{cases}
	\dfrac{2}{(1 + z) ^2} & \quad 0 \leq z \leq 1 \\
	0 & \quad ow \\
\end{cases}
$$
