به طور کلی می‌توانیم تشخص دهیم که تابع توزیع Z تنها برای مقادیرِ
$$
R_Z = \{1, 2, 3, 4\}
$$
ناصفر است. \\
حالا تابع توزیع Z را در هر یک از این مقادیر محاسبه می‌کنیم.\\
برای 
$
z = 1
$
داریم:
\begin{align*}
P_Z(1) &= \sum\limits_{y \in R_y}^{} P_X(1 - y)P_Y(y) \\
&= P_X(1 - 1)P_Y(1) + P_X(1 - 2)P_Y(2) \\
&= \frac{1}{3} \times \frac{1}{3} + 0 \times \frac{2}{3}\\
&= \frac{1}{9}
\end{align*}
\\
برای 
$ z = 2 $
داریم:
\begin{align*}
P_Z(2) &= \sum\limits_{y \in R_Y}^{} P_X(2 - y)P_Y(y) \\
&= P_X(2 - 1)P_Y(1) + P_X(2 - 2)P_Y(2) \\
&= \frac{1}{3} \times \frac{1}{3} + \frac{1}{3}\times\frac{2}{3}\\
&= \frac{1}{3} 
\end{align*}
در حالت 
$
z = 3
$
داریم:
\begin{align*}
P_Z(3) &= \sum\limits_{y \in R_Y} P_X(3 - y)P_Y(y) \\
&= P_X(3 - 1)P_Y(1) + P_X(3 - 2)P_Y(2) \\
&= \frac{1}{3}\times\frac{1}{3} + \frac{1}{3}\times\frac{2}{3}\\
&= \frac{1}{3} 
\end{align*}

و در حالت 
$ z = 4 $
داریم:
\begin{align*}
P_Z(4) &= \sum\limits{y \in R_Y} P_X(4 - y)P_Y(y) \\
&= P_X(4 - 1)P_Y(1) + P_X(4 - 2)P_Y(2)\\
&= 0\times\frac{1}{4} + \frac{1}{3}\times\frac{2}{3}\\
&= \frac{2}{9}
\end{align*}
بنابراین، تابع توزیع متغیر تصادفی Z عبارتست از:
$$
P_Z(z) =
\begin{cases}
\frac{1}{9} &\quad z = 1\\
\frac{1}{3} &\quad z = 2\\
\frac{1}{3} &\quad z = 3\\
\frac{2}{9} &\quad z = 4\\
0 &\quad \text{otherwise}
\end{cases}
$$
